\documentclass{article}
\usepackage{fontspec}
\usepackage{xcolor}
%\usepackage{sagetex}

\usepackage{euler}
\usepackage{amsmath}
\usepackage{amssymb}
\usepackage{unicode-math}


\usepackage[makeroom]{cancel}
\usepackage{ulem}

\setlength\parindent{0em}
\setlength\parskip{0.618em}
\usepackage[a4paper,lmargin=1in,rmargin=1in,tmargin=1in,bmargin=1in]{geometry}

\setmainfont[Mapping=tex-text]{Helvetica Neue LT Std 45 Light}

\newcommand\N{\mathbb{N}}
\newcommand\Z{\mathbb{Z}}
\newcommand\R{\mathbb{R}}
\newcommand\C{\mathbb{C}}
\newcommand\A{\mathbb{A}}

\usepackage[thinlines]{easytable}
\begin{document}

Q2: Let $A \in M_{2×2} (\C)$ and define a ring homomorphism $\phi : \C[x] → M_{2×2} (\C)$ by
\[\phi(f ) := f (A)\]
Find a quadratic polynomial $f (x)$ such that $\phi(f ) = 0$ (where $0$ means the matrix with all $0$ entries).
(You can assume the entries of $A$ are $a, b, c, d \in \C$, and you can write your answer in terms of these
entries if you want to.)

\uwave{pf.}

The characteristic polynomial of $A$ works.

Let $A = \begin{pmatrix}a & b\\c&d\end{pmatrix} \in M_{n\times
  n}(\C).$

Then $p(\lambda) = \lambda^2 -(a+d)\lambda +(ad-bc)I$.
\begin{align*}
  p(A) &= A^2 -(a+d)A +(ad-bc)I \\
       &= \begin{pmatrix}a & b\\c&d\end{pmatrix}
    -(a+d)\begin{pmatrix}a & b\\c&d\end{pmatrix}
    +(ad-bc)\begin{pmatrix}1 &0\\0&1\end{pmatrix}\\
       &= \begin{pmatrix}a^2 +bc & ab+bd\\ca+dc&cb+d^2\end{pmatrix}
                                                 +\begin{pmatrix}-a^2-ad &
                                     -ba-bd\\-ca-cd&-da-d^2\end{pmatrix}
                                           +\begin{pmatrix}ad-bc &
                                             0\\0&ad-bc\end{pmatrix}\\
       &=\begin{pmatrix}a^2 +bc -a^2-ad & ab+bd -ba-bd\\ca+dc-ca-cd&cb+d^2-da-d^2\end{pmatrix}
                                                +\begin{pmatrix}ad-bc &
                                                  0\\0&ad-bc\end{pmatrix}\\
       &= \begin{pmatrix} bc-ad & 0 \\0 &  cb-da\end{pmatrix}
         +\begin{pmatrix} ad-bc & 0 \\ 0 & ad-bc\end{pmatrix}\\
  &= \begin{pmatrix} 0&0\\0&0\end{pmatrix}
\end{align*}

That is $\phi(p) = 0$

\end{document}


%%% Local Variables:
%%% mode: latex
%%% TeX-master: t
%%% End:
