\documentclass{article}
\usepackage{fontspec}
\usepackage{xcolor}
%\usepackage{sagetex}

\usepackage{euler}
\usepackage{amsmath}
\usepackage{amssymb}
\usepackage{unicode-math}


\usepackage[makeroom]{cancel}
\usepackage{ulem}

\setlength\parindent{0em}
\setlength\parskip{0.618em}
\usepackage[a4paper,lmargin=1in,rmargin=1in,tmargin=1in,bmargin=1in]{geometry}

\setmainfont[Mapping=tex-text]{Helvetica Neue LT Std 45 Light}

\newcommand\N{\mathbb{N}}
\newcommand\Z{\mathbb{Z}}
\newcommand\Q{\mathbb{Q}}
\newcommand\R{\mathbb{R}}
\newcommand\C{\mathbb{C}}
\newcommand\A{\mathbb{A}}

\usepackage{harpoon}
\newcommand*{\harp}[1]{\overrightharp{\ensuremath{#1}}}


\usepackage[thinlines]{easytable}
\begin{document}

\paragraph{4} $\text{ }$

Here we consider properties of $1 − x$ viewed as an element of different rings.

(a) Show that $f = 1 − x$ is a unit in $\Z[[x]]$.

(b) Show that $f (x) = 1 − x$ is not a unit in the ring of continuous functions on $\R$.

(c) Show that $f (x) = 1 − x$ is a unit in the ring of continuous
functions on $[0, 1/2]$.

\uwave{slu. of (a)}
\[\text{Let } S = \sum_{n=0}^\infty x^n = 1 + \sum_{n=1}^\infty x^n\]
\[\implies xS = x\sum_{k=0}^\infty x^k = \sum_{k=0}^\infty x^{k+1} \]
\[ \text{Let } n = k+1 \implies xS = \sum_{n=1}^\infty x^n \]
\[ S =  1 + \sum_{n=1}^\infty x^n \implies S =
  1 + xS \implies S-xS = 1\]
\[\implies (1-x)\sum_{n=0}^{\infty}x^n = 1 \implies 1-x\text{ is a unit
    in }R[[x]]\quad \blacklozenge\]

\uwave{slu. of (b)}

$f$ would be a unit if there exists $g$ in $C(\R)$ where,

\[(fg)(x) = f(x)g(x) = 1\]

That is if $f(x) = 1-x$, then, \[(1-x)g(x) = 1 \iff g(x) =
  \frac{1}{1-x}\text{ .}\]

But, $g(x) \not\in C(\R),$ since $g(1)$ is an essential discontinuity
of $g$. That is we can't find some other function that is
continuous on $\R$, since neither the left hand limit nor the right
hand limit of $g$
as $x\rightarrow 0$ exist. So, $1-x$ is not a unit in the ring of
continuous functions on $\R\quad \lozenge$

\uwave{slu. of (c)}

$1\not\in [0,\frac{1}{2}]$ so $\frac{1}{1-x}$ is continuous on
$[0,\frac{1}{2}]$. So by the computation in the solution of (b) we
have that $1-x$ is a unit in $C([0,\frac{1}{2}])\quad \lozenge$

\end{document}


%%% Local Variables:
%%% mode: latex
%%% TeX-master: t
%%% End:
