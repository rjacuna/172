\documentclass{article}
\usepackage{fontspec}

% Used to embed Sage code in latex
\usepackage{sagetex}

% Math Environment
\usepackage{euler}        % Euler font
\usepackage{amsmath}      % Math macros
\usepackage{amssymb}      % Math symbols
\usepackage{unicode-math} % Unicode support


\usepackage[makeroom]{cancel} % Used to cancel terms in algebraic equations
\usepackage{ulem} % Different underline environments
\usepackage{polynom} %Polynomial long division

% Typesetting Rules
\setlength\parindent{0em}
\setlength\parskip{0.618em}
\usepackage[a4paper,lmargin=1in,rmargin=1in,tmargin=1in,bmargin=1in]{geometry}
\setmainfont[Mapping=tex-text]{Helvetica Neue LT Std 45 Light}

% Common Macros
\newcommand\N{\mathbb{N}}
\newcommand\Z{\mathbb{Z}}
\newcommand\Q{\mathbb{Q}}
\newcommand\R{\mathbb{R}}
\newcommand\C{\mathbb{C}}
\newcommand\A{\mathbb{A}}
\def\res{\mathop{\text{Res}}\limits}

% Color
\usepackage[dvipsnames]{xcolor}
\usepackage{pagecolor}
\definecolor{DeepMossGreen}{HTML}{394820}
\pagecolor{DeepMossGreen}
\color{Goldenrod}

\begin{document}

\begin{center}
  172 --- 5

  RJ Acuña

  (862079740)
\end{center}\vspace{1.618em}

\begin{enumerate}

\item Suppose that $F$ is a finite field with $q$ elements. Let $E
  \supset F$ be a field extension and suppose $\alpha \in E$ is
  algebraic over $F$ with degree $n$. Show that as a set, $F(\alpha)$
  has $q^n$ elements.

  \uwave{slu.}

    Since the degree of $\alpha$ over $F$ is $n,$ then $F(\alpha)$ is
    an $n-$dimensional vector space over $F$ with basis $\Alpha = \{\alpha^k\}_{k=0}^{n-1}$.

    Let, $a_k\in F,\, r\in F(\alpha)$, then $r = \sum_{k=0}^{n-1}a_k
    \alpha^k$.

    If $n=1$, then $F(\alpha) = F$, so $|F(\alpha)| = |F| = q$.

    If $n=2$, then $r= a_0 + a_1\alpha$. There are $q$ possible
    choices for, $a_0$ and $a_1$ respectively. So, there are $q^2$
    possible ways to express $r$ as a linear combination of $1$ and
    $\alpha$. So, $|F(\alpha)| = q^2$.

    So in general we can think of an element of $F(\alpha)$ as
    consisting of $n$-slots $1,\alpha,\dots,\alpha^{n-1}$, and $q$
    possible entries $a_k$. Thus there are at least $q^n$ possible ways to determine
    an element $r$ of $F(\alpha)$. Since, $\Alpha$ is a basis for
    $F(\alpha)$, $r$ is uniquely determined, so there are at most
    $q^n$ possible ways to determine $r$.

    So, $|F(\alpha)|= q^n\quad \blacksquare$

\item Let $F$ be the field $\Z/2\Z$. Find an irreducible polynomial in $F[x]$ of degree 3. Use this to construct a field extension of $F$ that contains 8 elements.

  \uwave{slu.}

  Let $f(x) \in (\Z/2\Z)[x]$ be of degree $3$. Then,
  \[f(x) = a_3x^3 + a_2 x^2 + a_1 x + a_0,\, a_i \in \Z/2\Z\]

  There are $2^4$ possibilities for $f(x)$, since $\Z/2\Z = \{0,1\}.$

  If $f(x)$ is reducible, then $f(x) = g(x)r(x)$.

  Since, deg $f$ = deg $g$ + deg $r$, WLOG assume  deg $g = 2$.

  Then the only two possibilities for $r$ are $r(x) = x$ or $r(x) = x+1$. So if $f$ is reducible, \[f(x)=
    xg(x) \text{ or } f(x) = (x-1)g(x)\]

  So $f(0)=0$ or $f(1) = 0$ respectively. If neither $f(0)$ nor $f(1)$
  are $0$, then $r(x)$ is not a factor of $f$, thus $f$ is
  irreducible.

  \[f(0) = a_0\text{ and } f(1) = \sum_{i=0}^3 a_i\]

  So, $a_0 \neq 0 \implies a_0 = 1\implies \sum_{i=0}^3a_i = 1
  +\sum_{i=1}^3 a_i$. Since deg $f =3 \implies a_3 \neq 0 \implies a_3
  = 1$. Therefore, $\sum_{i=0}^3a_i = 1
  +\sum_{i=1}^2 a_i + 1 = \sum_{i=1}^2 a_i \neq 0\implies a_1 = 0
  \text{ and }a_2 =1 \text{ or }a_1 = 1 \text{ and } a_2 = 0$.

  Thus $h(x) = x^3+x^2 +1$ and $k(x) = x^3 +x +1$ are irreducible.

  Since $k$ is irreducible, $\langle k(x) \rangle$ is maximal, thus $E
  = (\Z/2\Z)[x]/\langle k(x)\rangle$ is a field.

  Let $(\Z/2\Z)(\alpha)$ be a field extension  of $(\Z/2\Z)$ such that
  $k(\alpha)= 0$. Since the degree of $\alpha$ over $(\Z/2\Z)$ is
  equal to the degree of the irreducible polynomial that vanishes at
  $\alpha$, the degree of $\alpha$ over $(\Z/2\Z)$ is $3$. So, by the
  previous problem it has $2^3= 8$ elements$\quad \lozenge$
\newpage
\item Find a basis for the field $\Q(\sqrt 2 + \sqrt 5)$ as a vector space over $\Q$.

  \uwave{slu.}

  \begin{align*}
    \alpha = \sqrt{2}+ \sqrt{5}
    &\implies (\alpha - \sqrt{2})^2 =
      5\\
    &\implies  (\alpha - \sqrt{2})^2 - 5 = 0\\
    &\implies  \alpha^2 -2 \alpha\sqrt{2} +2 - 5 = 0\\
    &\implies  \alpha^2  - 3 = 2 \alpha\sqrt{2}\\
    &\implies  (\alpha^2  - 3)^2 = (2 \alpha\sqrt{2})^2\\
    &\implies  \alpha^4-6\alpha^2 +9 = 8 \alpha^2\\
    &\implies  \alpha^4-14\alpha^2 +9 = 0
  \end{align*}

  Thus $f(x) = x^4-14x^2 +9 \implies f(\sqrt{2}+\sqrt{5}) = 0$

  By the quadratic formula, $f(x) = 0$
  \[\implies x^2 = \frac{14 \pm\sqrt{196 -36}}{2} = \frac{14\pm
      \sqrt{160}}{2} = \frac{14\pm
      4\sqrt{10}}{2} = 7 \pm 2{\sqrt{10}}\]
  \[\implies x = \pm \sqrt{7 \pm 2{\sqrt{10}}}\]

    So, $f$ doesn't factor over $\Q$, thus $f$ is irreducible.

    Therefore $\{1, \sqrt{2}+\sqrt{5}, (\sqrt{2}+\sqrt{5})^2,
    (\sqrt{2}+\sqrt{5})^3\}$ is a basis for
    $\Q(\sqrt{2}+\sqrt{5})\quad \blacksquare$

\item Let $E \supset F$ be a field extension  and let $\alpha \in E$
  be algebraic over $F$ with odd degree. Show that $F(\alpha) =
  F(\alpha^2)$. Conversely, find an $\alpha \in \R$ which is algebraic
  over $\Q$ with even degree such that $\Q(\alpha) \not=
  \Q(\alpha^2)$.

  \uwave{slu.}

  Since $f(x) = x^2-\alpha^2 \in F(\alpha^2)[x] \implies f(\alpha) =
  0\implies [F(\alpha):F(\alpha^2)] = 2$. It follows that, \[[F(\alpha):F] =
    [F(\alpha):F(\alpha^2)] [F(\alpha^2):F] =2[F(\alpha^2):F]\]
  That
  contradicts that $[F(\alpha):F]$ is odd.

  $\alpha = \sqrt{3} \implies \alpha^2-3 = 0\implies$ deg $\alpha =
  2$, and $\Q(\sqrt{3})\neq \Q(3) = \Q$, since $3\in \Q \quad \blacksquare$

\end{enumerate}



\end{document}
%%% Local Variables:
%%% mode: latex
%%% TeX-master: t
%%% End:
