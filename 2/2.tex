\documentclass{article}
\usepackage{fontspec}

% Used to embed Sage code in latex
\usepackage{sagetex}

% Math Environment
\usepackage{euler}        % Euler font
\usepackage{amsmath}      % Math macros
\usepackage{amssymb}      % Math symbols
\usepackage{unicode-math} % Unicode support


\usepackage[makeroom]{cancel} % Used to cancel terms in algebraic equations
\usepackage{ulem} % Different underline environments
\usepackage{polynom} %Polynomial long division

% Typesetting Rules
\setlength\parindent{0em}
\setlength\parskip{0.618em}
\usepackage[a4paper,lmargin=1in,rmargin=1in,tmargin=1in,bmargin=1in]{geometry}
\setmainfont[Mapping=tex-text]{Helvetica Neue LT Std 45 Light}

% Common Macros
\newcommand\N{\mathbb{N}}
\newcommand\Z{\mathbb{Z}}
\newcommand\Q{\mathbb{Q}}
\newcommand\R{\mathbb{R}}
\newcommand\C{\mathbb{C}}
\newcommand\A{\mathbb{A}}
\def\res{\mathop{\text{Res}}\limits}

% Color
\usepackage[dvipsnames]{xcolor}
\usepackage{pagecolor}
\definecolor{DeepMossGreen}{HTML}{394820}
\pagecolor{DeepMossGreen}
\color{Goldenrod}

\begin{document}

\begin{center}
  172 --- 2

  RJ Acuña

  (862079740)
\end{center}\vspace{1.618em}


The \emph{center} of a ring $R$ is
\[ Z(R) := \{x \in R \mid xa=ax \quad \forall a \,\in\, R\}
\]
\begin{enumerate}
\item Show $Z(R)$ is a subring of $R$.

  \uwave{pf.}

  Let $x,y \in Z(R), a \in R$. Then,
  \[a0=0=0a\]
  \begin{align*}
    a,x,y \in R &\implies a(x-y) = ax-ay\\
    y\in Z(R) &\implies a(x-y) = ax-ya\\
    x\in Z(R) &\implies a(x-y) = xa-ya\\
    a,x,y \in R &\implies xa-ya = (x-y)a\\
    &\implies a(x-y) = (x-y)a
  \end{align*}
  \begin{align*}
    y\in R &\implies xy = yx\\
    \text{Multiplication by } a &\implies axy = ayx\\
    y\in Z(R) &\implies axy = yax\\
    x\in Z(R) &\implies axy = yxa
  \end{align*}
    So, $x,y\in \Z(R) \implies  0, x-y,$ and $ xy \in \Z(R)$.

  Thus, by the conclusion of exercise 48  in p. 176
  $Z(R)$ is a sub-ring of $R$ $\quad \blacksquare$

\item Show that the center of $M_{2\times 2}(\R)$ is spanned (as a vector space) by the identity matrix.

  \uwave{pf.}

  Suppose, $\exists X=\begin{pmatrix} x_1 & x_2\\x_3 &
    x_4 \end{pmatrix} \in Z\left( M_{2\times 2}(\R) \right): \forall
  r\in \R \quad X\neq
  rI$. Let $A = \begin{pmatrix} 1&0\\0&0\end{pmatrix}$ and $B =\begin{pmatrix} 0&1\\0&0\end{pmatrix}$
  \begin{align*} XA = AX \implies \begin{pmatrix} x_1 & x_2\\x_3 &
      x_4 \end{pmatrix} \begin{pmatrix} 1&0\\0&0\end{pmatrix} &=
  \begin{pmatrix} x_1&0\\x_3&0\end{pmatrix}\\ &= \begin{pmatrix} x_1
    &x_2\\0&0\end{pmatrix}\\ &= \begin{pmatrix} 1&0\\0&0\end{pmatrix} \begin{pmatrix} x_1 & x_2\\x_3 &
      x_4 \end{pmatrix}\\&\implies
    x_2=x_3 = 0&\implies X = \begin{pmatrix}x_1 &0 \\ 0&x_4\end{pmatrix} \end{align*}
  \begin{align*} XB = BX \implies \begin{pmatrix} x_1 & 0\\0 &
      x_4 \end{pmatrix} \begin{pmatrix} 0&1\\0&0\end{pmatrix} &=
  \begin{pmatrix} 0&x_1\\0&0\end{pmatrix}\\ &= \begin{pmatrix} 0
    &x_4\\0&0\end{pmatrix}\\ &= \begin{pmatrix}
    0&1\\0&0\end{pmatrix} \begin{pmatrix} x_1 & 0\\0 &
      x_4 \end{pmatrix}\\&\implies
    x_1=x_4\\ &\implies X = x_1I \end{align*}
  $X\neq rI \rightarrow\leftarrow X = x_1I$. So by contradiction $X =
  rI$.

  Therefore, $Z(M_{2\times 2}(\R))=$ Span $\{I\}\quad \blacksquare$
\item Let $R = \R S_n$ be the group algebra of the symmetric group, and define the element
\[
x = \sum_{1 \leq i \not= j \leq n} (i\,j)
\]
(For example, if $n=3$, then $x = (1\,2) + (1\,3) + (2\,3)$.) Show that for any transposition $(k\,\ell) \in S_n$ we have $(k\,\ell) x (k\,\ell) = x$.\\
(Hint: you can use the fact that if $\sigma \in S_n$ and $(a_1 \, \cdots a_k)$ is a cycle in $S_n$, then  $\sigma (a_1 \,a_2 \cdots \, a_k)\sigma^{-1} = (\sigma(a_1) \, \cdots \sigma(a_k))$.)\\
Note: this actually shows $x \in Z(S_n)$, because any element of $S_n$ is a product of transpositions.

\uwave{pf.}

\[
(k\,l)x(k\,l)= \sum_{1 \leq i \not= j \leq n} (k\,l)(i\,j)(k\,l)
\quad\text{ by
  left and right distribution}
\]

Now either, (I) $i\neq k$ and $j\neq k$, (II)  $i = k$ and $j\neq l$, (III)$i \neq k$
and $j= l$, or (IV) $i=k$ and $j=l$.

(I): $(k\, l)(i\,j)(k\, l) = (k\, l)(k\, l)(i\, j) = (i\, j)$

(II): $(k\, l)(k\,j)(k\, l) = (k\, l)(l\, j) = (k\, j)$

(III): $(k\, l)(i\,l)(k\, l) = (k\, l)(k\, i) = (i\, l)$

(I): $(k\, l)(k\,l)(k\, l) = (k\, l)(k\, l)(k\, l) = (k\, l)$

Thus, \[(k\,l)x(k\,l) = x\quad \blacksquare\]
\item Use Fermat's theorem to compute the remainder of $37^{49}$ when it is divided by $7$.

  \[\text{\uwave{slu.} }37^{49} = 37^{48+1} = 37^{48}37 = 37^{6\cdot 8}37= (37^6)^8 37
    \equiv_7 1^8 37 \equiv_7 37 \equiv_7 5\cdot 7 +2 \equiv_7 2\text{
      So, 2}\quad \blacklozenge\]


\item Let $\phi_a: R[x] \to R$ be the evaluation homomorphism, defined
  by $\phi_a(f) = f(a)$.

  Let $f(x) = (x-2)(x+3)$.
\begin{enumerate}
\item If $R = \Z$, find all $a \in R$ with $\phi_a(f) = 0$.

  \uwave{slu.} $\Z$ is an integral domain, thus $a= 2$, or $a =
  -3\quad \lozenge$

\item If $R = \Z/8\Z$, find all $a \in R$ with $\phi_a(f) = 0$.

  \uwave{slu.}
  \begin{align*}
    f(1) = (1-2)(1+3) &\equiv_8 4\\
    f(2) = (2-2)(2+3) &\equiv_8 0\\
    f(3) = (3-2)(3+3) &\equiv_8 6\\
    f(4) = (4-2)(4+3) &\equiv_8 6 \\
    f(5) = (5-2)(5+3) &\equiv_8 0\\
    f(6) = (6-2)(6+3) &\equiv_8 4\\
    f(7) = (7-2)(7+3) &\equiv_8 2\\
  \end{align*}
  So, $a = 2$ or $a = 5\quad \lozenge$
\end{enumerate}


\end{enumerate}

\end{document}
%%% Local Variables:
%%% mode: latex
%%% TeX-master: t
%%% End:
