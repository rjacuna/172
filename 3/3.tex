\documentclass{article}
\usepackage{fontspec}

% Used to embed Sage code in latex
%\usepackage{sagetex}

% Math Environment
\usepackage{euler}        % Euler font
\usepackage{amsmath}      % Math macros
\usepackage{amssymb}      % Math symbols
\usepackage{unicode-math} % Unicode support


\usepackage[makeroom]{cancel} % Used to cancel terms in algebraic equations
\usepackage{ulem} % Different underline environments
\usepackage{polynom} %Polynomial long division

% Typesetting Rules
\setlength\parindent{0em}
\setlength\parskip{0.618em}
\usepackage[a4paper,lmargin=1in,rmargin=1in,tmargin=1in,bmargin=1in]{geometry}
\setmainfont[Mapping=tex-text]{Helvetica Neue LT Std 45 Light}

% Common Macros
\newcommand\N{\mathbb{N}}
\newcommand\Z{\mathbb{Z}}
\newcommand\Q{\mathbb{Q}}
\newcommand\R{\mathbb{R}}
\newcommand\C{\mathbb{C}}
\newcommand\A{\mathbb{A}}
\def\res{\mathop{\text{Res}}\limits}

% Color
\usepackage[dvipsnames]{xcolor}
\usepackage{pagecolor}
\definecolor{DeepMossGreen}{HTML}{394820}
\pagecolor{DeepMossGreen}
\color{Goldenrod}

\begin{document}

\begin{center}
  172 --- 3

  RJ Acuña

  (862079740)
\end{center}\vspace{1.618em}


\begin{enumerate}
\item Let $R$ be a commutative ring with identity element $1_R \in R$, and suppose that $R$ has characteristic $p \in \Z$. (I.e. $pa = 0$ for all $a \in R$.) Show that the map $\phi:R \to R$ defined by $\phi(a) = a^p$ is a ring homomorphism.

  \uwave{pf.}

  Let $a,b\in R$
\begin{align*}
  \phi(a+b)
  &= (a+b)^p\\
  &= \sum_{k=0}^p \begin{pmatrix}p \\
    k\end{pmatrix} a^kb^{p-k}\\
  &= \sum_{k=0}^p
    \frac{p!}{k!(p-k)!}a^kb^{p-k}\\
  &= \frac{p!}{0!(p-0)!}a^0b^{p-0} + \sum_{k=1}^{p-1}
  \frac{(p-1)!}{k!(p-k)!} pa a^{k-1}b^{p-k} +
    \frac{p!}{p!(p-p)!}a^pb^{p-p}\\
  &= b^{p} + \sum_{k=1}^{p-1}
    \frac{(p-1)!}{k!(p-k)!} 0 a^{k-1}b^{p-k} + a^p \text{ by } pa=0\\
  &= b^p+a^p\\
  &= a^p + b^p\\
  &=\phi(a)+\phi(b)\\
  \phi(ab)
  &= (ab)^p\\
  &= abab\cdots abab\quad\, \text{ p times} \\
  &= a^pb^p\hspace{5em} \text{by commutativity of multiplication}\\
  &= \phi(a)\phi(b)
\end{align*}
So, $\phi$ is a ring homomorphism$\quad \blacksquare$
\item Let $R$ be a commutative ring and let $a \in R$. Show that $I_a := \{x \in R \,\mid\, ax=0\}$ is an ideal of $R$.

  \uwave{pf.}

  Let $y\in R$, and $x \in I_a$.
  \[ yxa =yax= y0 = 0 = 0y = axy\quad \text{by commutativity of } R \]
  Shows, that any  $x \in I_a$ and $y$ is any element of $R$, then both $xy,
  yx\in I_a$ so $I_a$ is an ideal $\quad \blacksquare$
\item Let $R$ be a commutative ring and $I \subset R$ an ideal. Show that the following set is also an ideal of $R$:
\[
\sqrt{I} := \{a \in R \,\mid\, a^k \in I \textrm{ for some } k\}
\]

\uwave{pf.}

Let $a\in \sqrt{I}$, and $b\in R$.

$a\in \sqrt{I}\implies \exists k\in\Z : a^k\in I$, then $(ab)^k = a^kb^k= (ba)^k$ by the commutativity of $R$.

Since $b^k\in \R$, then $a^kb^k\in I$, so $(ab)^k$ and $(ba)^k$ are in
$I$.

Therefore $ab$ and $ba$ are in $\sqrt{I}$ and $\sqrt{I}$ is an
ideal $\quad \blacksquare$
\newpage
\item Find $\sqrt{I}$ for the following two ideals:
\begin{enumerate}
\item $I = 500\Z \subset \Z$

  \uwave{slu.}

  $\forall m\in \Z:\exists n,k\in \Z: m^k = 500n $

  $\implies m = (500n)^{1/k}= (5\cdot 10^2 n)^{1/k} = (5^3\cdot2^2
  n)^{1/k}  $

  $k = 0$ doesn't work because $1\neq 500$.

  $k=1$ $\implies n =1\implies m=500$.

  $k = 2$, and $\implies m = (5^3\cdot2^2n )^{1/2}$

  We want to find $n$ such that the equation is solvable over the
  integers.

  The smallest such solutions would be desirable.

  $n = 5 \implies m = (5^4\cdot2^2 )^{1/2} = 50$

  $n = 5\cdot2^2 \implies m = (5^4\cdot2^4 )^{1/2} = 100$

  So, $50$ and $100$ are in $\sqrt{I}$.

  Now for, $k>2$ we can see the following,
  $k = 3$, and $n = 2\implies m = (5^3\cdot2^3)^{1/3} = 10$

  $k = 4$, and $n = 5\cdot2^2 \implies m =(5^4\cdot2^4)^{1/4} = 10$.

  $\forall l\in \Z: l\geq 3$ if $k = l$, we can put $n = 5^{l-3}\cdot 2^{l-2}$, when we have $m = (5^3\cdot2^2
  \cdot 5^{l-3}\cdot 2^{l-2})^{1/k}=
  10$.

  So $m$ is $500,100,50,$ or $10$. An note $10$ divides all the
  others.

  Thus, $(10l)^k = 10^kl^k = 500nl^k$ for some $k$. Therefore $10\Z
  \subset \sqrt{500\Z}$.

  Suppose for a contradiction $\exists m \in \sqrt{500\Z}: \forall l\in\Z: m \neq 10l$


  Since $m\in \sqrt{500\Z}$ we can see that there is a $k\in \Z$ such
  that,

  $m^k = 500n = 10\cdot 50n \implies 10|m^k \implies 10| m \implies
  \exists l\in \Z: m = 10l \rightarrow\leftarrow m\neq 10 l$

  Therefore $\sqrt{500\Z}\subset 10\Z$. And by the previous
  containment we can see that \[10\Z = \sqrt{500\Z}\quad \blacksquare\]
	\item $I = \langle x^3\rangle \subset \Q[x]$. \\
	Remember $\langle f(x) \rangle$ means ``the ideal generated by
        $f(x)$,'' so \[\langle f(x) \rangle = \{f(x)g(x)\,\mid\,
          g(x)\in \Q[x]\}\]

        \uwave{slu.}

        $\langle x \rangle\subset \sqrt{I}$ since given $x^n\in\langle x
        \rangle$, $(x^n)^3 = x^{3n}= x^3x^{3n-3}\in I,\,\forall
        n\in\Z_{\geq 0}$.

        Suppose for the sake of contradiction that
        $\exists f(x) \in \sqrt{I}:\forall
        g(x) \in \Q[x]: f(x) \neq x g(x)$

        $f(x) \in \sqrt{I} \implies \exists k\in \Z: f(x)^k = h(x)\in
        I\implies f(x)^k = x^3 d(x) = x x^2 d(x)\implies x|f(x)^k
        \implies x|f(x)$

        Therefore, we arrive at a contradiction. That is if
        $f(x)\in\sqrt{I},$ then $f(x) \in \langle x \rangle$.

        Thus $\sqrt{I}\subset \langle x \rangle$,\[\sqrt{I} = \langle
          x \rangle\quad \blacksquare\]


\end{enumerate}


\end{enumerate}


\end{document}
%%% Local Variables:
%%% mode: latex
%%% TeX-master: t
%%% End:
