\documentclass{article}
\usepackage{fontspec}

% Used to embed Sage code in latex
%\usepackage{sagetex}

% Math Environment
\usepackage{euler}        % Euler font
\usepackage{amsmath}      % Math macros
\usepackage{amssymb}      % Math symbols
\usepackage{unicode-math} % Unicode support

% Physics Environment
%\usepackage{physics}


\usepackage[makeroom]{cancel} % Used to cancel terms in algebraic equations
\usepackage{ulem} % Different underline environments
\usepackage{polynom} %Polynomial long division

% Typesetting Rules
\setlength\parindent{0em}
\setlength\parskip{0.618em}
\usepackage[a4paper,lmargin=1in,rmargin=1in,tmargin=1in,bmargin=1in]{geometry}
\setmainfont[Mapping=tex-text]{Helvetica Neue LT Std 45 Light}

% Common Macros
\newcommand\N{\mathbb{N}}
\newcommand\Z{\mathbb{Z}}
\newcommand\Q{\mathbb{Q}}
\newcommand\R{\mathbb{R}}
\newcommand\C{\mathbb{C}}
\newcommand\A{\mathbb{A}}
\def\res{\mathop{\text{Res}}\limits}

% Color
\usepackage[dvipsnames]{xcolor}
\usepackage{pagecolor}
\definecolor{Sea}{HTML}{219096}
\pagecolor{Sea}
\color{Goldenrod}

\begin{document}

\begin{center}
  172 --- 8

  RJ Acuña

  (862079740)
\end{center}\vspace{1.618em}



\begin{enumerate}

\item (Book 50.17) Let $E$ be a finite extension of $F$. Show that if $E$ is a splitting field, then it is a splitting field of a single polynomial $f(x) \in F[x]$.

  \uwave{pf.}

  If the field $F$ is finite of characteristic $p$ a prime number, then every finite extension of $F$ is
  generated by the polynomial $x^{p^n}-x$ for some $n$.

  Otherwise, for $F$ is infinite we have.

  Let $F(\alpha,\beta)$, be a finite extension of $F$.

  Let $m_{\alpha}$, and $m_{\beta}$ be the minimal polynomials
  for $\alpha$ and $\beta$.

  Let $\alpha_1,\dots, \alpha_n$, be the $n$ distinct roots of $m_{\alpha}$.

  Let $\beta_1,\dots, \beta_n$, be the $m$ distinct roots of $m_{\beta}$.

  Since, $F$ is infinite, let $\gamma = \alpha + \kappa\beta : \gamma
  \neq \alpha_i +\kappa \beta_j$ for $1\leq i\leq n$, and $1\leq j\leq
  m.$

  $h(x) := m_{\alpha}(\gamma - \kappa x)$.

  $h(\beta) = m_{\alpha}(\gamma - \kappa \beta) =  \m_{\alpha}(\alpha)
  = 0 \implies (x-\beta)| h.$

  So $h$ splits in $F(\gamma),$ and $\beta \in F(\gamma)$. But $\gamma
 -\kappa \beta = \alpha$, so $\alpha \in F(\gamma).$ So, $F(\gamma) =
 F(\alpha,\beta)$.

 Then an induction establishes the result $\quad \blacksquare$

\item (Book 50.20) Show that the automorphism group $Aut(\Q(2^{(1/3)})/ \Q)$ is the trivial group. (Remember that $Aut(E/F)$ is the group of automorphisms of $\varphi:E \to E$ such that $\varphi(a) = a$ for all $a \in F$.)

  \uwave{slu.}

  $x = 2^{1/3} \implies x^3 = 2 \implies x^3-2 = 0$.

  $x^3 - 2$ is thus the minimal irreducible polynomial for $2^{1/3}$,
  by Einsenstein's criterion with the prime being $2$.

  $x^3-2 = 0 \implies x^3 = 2\exp(i(2n\pi)) \implies x =
  2^{1/3}\exp(i\left(\frac{2n\pi}{3}\right))\quad (n = 0,1
  2)$.

  So, $x= 2^{1/3}$, or $x =
  2^{1/3}\exp(i\left(\frac{2\pi}{3}\right))$, or $x = 2^{1/3}\exp(i\left(\frac{4\pi}{3}\right))$.

  $\Q(2^{1/3})$, only contains one of the roots of $x^3-2$, so since
  permutations in $Aut(\Q(2^{(1/3)} / \Q)$ fix $\Q$, any such
  permutation must map $2^{1/3}\mapsto 2^{1/3}$. Thus,
  $Aut(\Q(2^{(1/3)} / \Q)$, is trivial$\quad \lozenge$

\item (Book 51.3) Find an $\alpha \in \Q(\sqrt{2}, \sqrt{3}) =: E$ so that $E = \Q(\alpha)$. Compute explicitly how to write $\sqrt 2$ and $\sqrt 3$ in terms of $\alpha$.

  \uwave{slu.}
  \begin{align*} \alpha = \sqrt{2} +\sqrt{3}
    &\iff \alpha-\sqrt{2} = \sqrt{3}\\
    &\iff \alpha^2-2\sqrt{2}\alpha +2 = 3\\
    &\iff \alpha^2  -1= 2\sqrt{2}\alpha\\
    &\iff \alpha^4 -2\alpha^2 +1= 8\alpha^2\\
    &\iff \alpha^4 -10\alpha^2 +1= 0
  \end{align*}
  By the quadratic formula,
  \[\alpha^2 = \frac{10 \pm \sqrt{96}}{2} = 5 \pm 2\sqrt{6} \implies
    \alpha = \pm \sqrt{5 \pm 2\sqrt{6}}\not\in \Q\]
  So, $x^4-10x^2 +1$ is irreducible over $\Q$. Furthermore,
  $\Q(\alpha)$ has basis $\{1,\alpha,\alpha^2,\alpha^3\}$.

  $\alpha^2 = 5+2\sqrt{6}$, $\alpha^3 = 11\sqrt{2}+9\sqrt{3}$.

  So $\sqrt{2} = \frac{9\alpha -\alpha^3}{9-11}$, and $\sqrt{3} =
  \frac{11\alpha -\alpha^3}{11-9}$. So $\Q(\alpha) =
  \Q(\sqrt{2},\sqrt{3})\quad \lozenge$

\item (Book 51.17) Let $K$ be a finite normal extension of $F$ and let $G = Aut(K/F)$. Define the \emph{norm} and \emph{trace} of an element $\alpha \in K$ as follows:
\[
N(\alpha) := \prod_{\sigma \in G} \sigma(\alpha),\quad \quad Tr(\alpha) := \sum_{\sigma \in G} \sigma(\alpha)
\]
Show that $N(\alpha)$ and $Tr(\alpha)$ are elements of $F$. (Hint: what is $\sigma(N(\alpha))$ for $\sigma \in G$? Then, look at part 2 of Theorem 53.6 in the special case $E = F$.)

\uwave{pf.}

Let $\tau \in G,\alpha \in K$. Then, since $\tau$ is a ring
homomorphism,
\[\tau(N(\alpha)) = N(\alpha),\text{ and } \tau(Tr(\alpha)) =
  Tr(\alpha).\]
So, since $\tau$ fixes $N(\alpha)$, and $Tr(\alpha)$,it follows that
they must be in $F\quad\lozenge$

\item (Book 51.18 (a,b,e,f)) Let $K = \Q(\sqrt 2, \sqrt 3)$. Using the definition in the previous problem, compute the norm and trace of $\alpha = \sqrt 2$ and $\beta = \sqrt 2 + \sqrt 3$.

  \uwave{slu.}

  By the definitions in page 450 $Aut(K/\Q) =
  \{\iota,\sigma_1,\sigma_2,\sigma_3\}$, with the following
  assignments leaving other elements fixed,
  \[\sigma_1; \sqrt{2} \mapsto -\sqrt{2},\sqrt{6}\mapsto -\sqrt{6}, \]
  \[\sigma_2; \sqrt{3} \mapsto -\sqrt{3},\sqrt{6}\mapsto -\sqrt{6}, \]
  \[\sigma_3; \sqrt{2} \mapsto -\sqrt{2},\sqrt{3}\mapsto -\sqrt{3}. \]

  So, \[N(\alpha) =
  \iota(\sqrt{2})\sigma_1(\sqrt{2})\sigma_2(\sqrt{2})\sigma_3(\sqrt{2}))
  = (\sqrt{2})(-\sqrt{2})(\sqrt{2})(-\sqrt{2}) = 4,\text{ and
  }Tr(\alpha) = 0.\]

So, \begin{align*} N(\beta) &=
  \iota(\sqrt{2}+\sqrt{3})\sigma_1(\sqrt{2}+\sqrt{3})\sigma_2(\sqrt{2}+\sqrt{3})\sigma_3(\sqrt{2}+\sqrt{3}))\\
  &=
    (\sqrt{2}+\sqrt{3})(-\sqrt{2}+\sqrt{3})(\sqrt{2}-\sqrt{3})(-\sqrt{2}-\sqrt{3})\\
                            &= -(5+2\sqrt{6})(-\sqrt{2}+\sqrt{3})(\sqrt{2}-\sqrt{3})\\
                            &=
                              -(5+2\sqrt{6})(-(2-\sqrt{6})+(\sqrt{6}-3))\\
                            &= -(5+2\sqrt{6})(-5 +2\sqrt{6})\\
                            &= -(-25+24) = 1, \text{ and } Tr(\beta) = 0.
    \end{align*}

\end{enumerate}


\end{document}
%%% Local Variables:
%%% mode: latex
%%% TeX-master: t
%%% End:
