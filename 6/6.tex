\documentclass{article}
\usepackage{fontspec}

% Used to embed Sage code in latex
%\usepackage{sagetex}

% Math Environment
\usepackage{euler}        % Euler font
\usepackage{amsmath}      % Math macros
\usepackage{amssymb}      % Math symbols
\usepackage{unicode-math} % Unicode support

% Physics Environment
\usepackage{physics}


\usepackage[makeroom]{cancel} % Used to cancel terms in algebraic equations
\usepackage{ulem} % Different underline environments
\usepackage{polynom} %Polynomial long division

% Typesetting Rules
\setlength\parindent{0em}
\setlength\parskip{0.618em}
\usepackage[a4paper,lmargin=1in,rmargin=1in,tmargin=1in,bmargin=1in]{geometry}
\setmainfont[Mapping=tex-text]{Helvetica Neue LT Std 45 Light}

% Common Macros
\newcommand\N{\mathbb{N}}
\newcommand\Z{\mathbb{Z}}
\newcommand\Q{\mathbb{Q}}
\newcommand\R{\mathbb{R}}
\newcommand\C{\mathbb{C}}
\newcommand\A{\mathbb{A}}
\def\res{\mathop{\text{Res}}\limits}

% Color
\usepackage[dvipsnames]{xcolor}
\usepackage{pagecolor}
\definecolor{Sea}{HTML}{219096}
\pagecolor{Sea}
\color{Goldenrod}

\begin{document}

\begin{center}
  172 --- 6

  RJ Acuña

  (862079740)
\end{center}\vspace{1.618em}


\begin{enumerate}

\item Recall that $\C(t) := \{f(t)/g(t) \,\mid\, f,g \in \C[t]\,\,g\not=0\}$ is the field of rational functions in a variable $t$ with coefficients in $\C$. Show that $\C(t)$ is not algebraically closed.

  \uwave{pf.}

  Consider $X^2 + t \in \C(t)[X],$
  Suppose for a contradiction that $f/g \in \C(t)$ solves
  \begin{equation}X^2+t = 0\end{equation}

  It follows that, \[\left(\frac{f}{g}\right)^2 +t = 0 \iff
    \frac{f^2}{g^2} + t = 0 \iff  f^2 + tg^2 = 0 \iff f^2 = -tg^2
    \iff -t \mid f^2\]
  \[f^2 = f\cdot f \text{ and } -t \mid f^2 \implies -t\mid f
    \implies \exists t\in \C[t]: f = -t h\]
  \[\implies (-th)^2 = -tg^2 \iff (-th)^2 + tg^2 = 0 \iff t^2h^2 +tg^2
    = 0 \iff t(th^2 +g^2) = 0\]
  \[\implies g^2 + th^2 = 0 \implies -t\mid g \implies \exists k\in
    \C[t]: g = -tk\]
  Then $g(0) = -0k$,  so $g\neq 0$ is false. Therefore, $f/g \not\in
  \C(t)\rightarrow\leftarrow$

  So, $\C(t)$ is not algebraically closed$\quad \blacksquare$
 \item Suppose that $F \subset E$ is a \emph{finite} field extension
   (so $E$ is finite dimensional as a vector space over $F$). Show
   that if $F \subset R \subset E$ and $R$ is a ring, then $R$ is a
   field.


   \uwave{pf.}

   Let $F$ be a field, and $F \subset E$ be a \emph{finite} field
   extension. Then $E$ is a finite dimensional vector space over
   $F$ of dimension $n$. Let $\{\vb{e}_i\}_{i=1}^{n}$ be a basis for $E$.

   $R\subset E \implies \forall \vb{r}\in R,\exists! t_i\in F:  \vb{r} =
   \sum_{i=1}^n t_i\vb{e}_i$,

   If $R = E$, we're done,

   Notice, $F\subset R\implies 0\in R$, and since $R$ is a ring it is
   closed under addition.

   If $R \neq E$ then, $\exists \vb{x}\in E: \vb{x}\not\in R$, Let $\langle \vb{x} \rangle$ be the span of $\vb{x}$.

   $\forall s\in F, s\vb{x}\in \langle \vb{x} \rangle$. Suppose $\exists s\in F: s\vb{x} \in R$, then
   \[F\subset R\text{ and } R\text{ is a ring} \implies \frac{1}{s}
     \in R \text{ and } \frac{1}{s}s\vb{x} = \vb{x}
     \in R \text{ respectively.}\]
   That contradicts that $\vb{x}\not\in E$. So $\langle \vb{x}
   \rangle\cap R = \{0\}$

   Since $\{\vb{x}\}$ is a basis for $\langle \vb{x} \rangle$, and
   since dim$_F E = n$ we can
   extend it to a basis$\{\vb{x}\} \cup \{ \vb{u}_i\}_{i=1}^{n-1}$ for $E$.

   We've shown that $R \subset \langle \vb{u}_1,\cdots, \vb{u}_{n-1}
   \rangle$. Now, if $R = \langle \vb{u}_1,\cdots, \vb{u}_{n-1} \rangle$, then
   $R$ is a field, as it is a vector subspace of a
   finite field extension. Otherwise, we can find a $\vb{y} \in
   \langle \vb{u}_1,\cdots, \vb{u}_{n-1} \rangle$, such that
   $\{\vb{y}\}\cup\{ \vb{w}_i\}_{i=1}^{n-2}$ is a basis of $\langle
   \vb{u}_1,\cdots, \vb{u}_{n-1} \rangle$, and $R\subset
   \langle \vb{u}_1,\cdots, \vb{u}_{n-1} \rangle$. This process is
   finite, since the dimension is $n$.

   After $m$ steps there is a minimal basis $B
   =\{\vb{v}_i\}_{i=1}^m,$ where $m$ satisfies $
   1 \leq m < n-1$, such that
   $R\subset \langle B \rangle$.
   And, it also satisfies that if we remove any element $\vb{v}_k$ of
   the basis $B$, then
  $\langle B\backslash\{\vb{v}_k\} \rangle \subset R$.

   Let $r\in R$ such that $r \not\in \langle B\backslash\{\vb{v}_k\}
   \rangle$, then $F\subset R \implies \exists! t \in F: r =
   t\vb{v}_k$, since $R \subset \langle  B \rangle$. So, if there
   exists such $r$, it follows that $R = \langle B \rangle$. If, there
   isn't such an $r$, then $R = \langle
   B\backslash\{\vb{v}_k\}\rangle$, contradicting the minimality of
   $B$. Therefore $R$ is a finite dimensional vector space over
   $F$. So, $R$ is a field$\quad \blacksquare$

\item Let $R$ be the smallest subring of $\R$ containing $\Q$ and
  $\pi$, so that $\Q \subset R \subset \R$. Show that $R$ is not a
  field by showing $\pi^{-1}$ is not in $R$.

  \uwave{pf.} Since $\pi$, is a
  transcendental number. It follows that $\forall f(x)\in \Q[x],
  f(\pi)\neq 0$. Suppose $\pi^{-1} \in \Q[\pi],$ then for some $f(x) =
  \sum_{k=1}^n a_kx^k \in \Q[x]$,  \[\pi^{-1} = \sum_{k=0}^n
    a_k\pi^k\iff  1 = \sum_{k=0}^{n}
    a_k\pi^{k+1} \iff 0 = -1 +\sum_{k=0}^{n}
    a_k\pi^{k+1} = h(\pi) \]
\[\implies \exists h(x)\in \Q[x]: h(\pi) = 0
    \rightarrow\leftarrow \]
  $\pi^{-1}\not\in R= \Q[\pi] \implies R$ is not a field $\quad \blacksquare$

\item Let $F$ be a finite field with $p^n$ elements (which we showed in class is automatically a field extension of $\Z/p\Z$). Suppose that $\alpha \in F$ generates the group $F^\times$ of units in $F$. Show that $deg(\alpha; \Z/p) = n$.

\uwave{pf.}

By 33.11 for every finite field $\Z/p\Z$, and for every $n\in \N$ there exists
an irreducible polynomial $f(x)$ of
degree $n$ in $(\Z/p\Z)[x]$. Then, since $f(x)$ is irreducible, $\langle f(x) \rangle$ is maximal,
therefore $(\Z/p\Z)[x]/\langle f(x) \rangle$ is a field. Furthermore,
it has $p^n$ elements, sine it is an $n$-dimensional vector space over
$\Z/p\Z$, which has $p$ elements. By 33.12 $(\Z/p\Z)[x]/\langle f(x)
\rangle$ is  isomorphic to $F$. So,
deg$(\alpha;\Z/p\Z) =$ deg$f = n\quad \blacksquare$

\end{enumerate}



\end{document}
%%% Local Variables:
%%% mode: latex
%%% TeX-master: t
%%% End:
