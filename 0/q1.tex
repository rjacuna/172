\documentclass{article}
\usepackage{fontspec}
\usepackage{xcolor}
%\usepackage{sagetex}

\usepackage{euler}
\usepackage{amsmath}
\usepackage{amssymb}
\usepackage{unicode-math}


\usepackage[makeroom]{cancel}
\usepackage{ulem}

\setlength\parindent{0em}
\setlength\parskip{0.618em}
\usepackage[a4paper,lmargin=1in,rmargin=1in,tmargin=1in,bmargin=1in]{geometry}

\setmainfont[Mapping=tex-text]{Helvetica Neue LT Std 45 Light}

\newcommand\N{\mathbb{N}}
\newcommand\Z{\mathbb{Z}}
\newcommand\R{\mathbb{R}}
\newcommand\C{\mathbb{C}}
\newcommand\A{\mathbb{A}}

\usepackage[thinlines]{easytable}
\begin{document}

Q1:

Did you take math 171?

Yes.

Have you used LaTex before for typing mathematics?

Yes.

Do you have a physical or electronic copy of the book?

Yes.


\end{document}


%%% Local Variables:
%%% mode: latex
%%% TeX-master: t
%%% End:
